\chapter{Modèles}

\section{Modèles investigués}

\subsection{YOLOv5}
\paragraph{Description}

\subsection{FASTER R-CNN}

source1 : https://towardsdatascience.com/r-cnn-fast-r-cnn-faster-r-cnn-yolo-object-detection-algorithms-36d53571365e \newline\newline
source2 : https://stackoverflow.com/questions/48318448/understanding-basic-difference-between-a-cnn-and-rnn \newline\newline
+ check videos en cherchant "rcnn, fast rcnn, faster rcnn" sur google






\paragraph{Description}





\begin{itemize}
    \item Histoire (evolution de RCNN) 
    \item Date du papier l’expliquant 
    \item architecture
    \item Score benchmark coco
\end{itemize}
Adapter les exemples sur un problèmes légèrement différent : bullshiter sur assembler différent tutoriaux nécessite de comprendre, perte de temps sur manque de connaissance du jargon technique : mAP, etc les benchmarks coco, comprendre que coco (Common Object in COntext) est beaucoup de chose (dataset, benchmark baseline, dataset d’entrainement)





\subsection{RETINA net}
\paragraph{Description}

\subsection{DEtection Transformer (DE-TR)}
\paragraph{Description}

% ----------------------------------------------------

\section{Choix des modèles}

\subsection{Model1}
\paragraph{Description}
\paragraph{Training}
\paragraph{Results}

\subsection{Model2}
\paragraph{Description}
\paragraph{Training}
\paragraph{Results}