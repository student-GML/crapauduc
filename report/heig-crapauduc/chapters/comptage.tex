\chapter{Comptage}

\paragraph Le comptage des animaux traversant le tunel est la tâche principale pour laquelle ce projet a été mis sur pieds. Maintenant que nous avons des résultats probants pour l'object detection, l'étape suivante consiste en la mise en oeuvre d'une stratégie permettant de compter le nombre de tritons et grenouilles-crapauds traversant le tunnel. Ceci est une tâche assez complexe dans la mesure où une fois les capteurs enclenchés, une longue serie d'images est prise. De ce fait plusieurs images par exemple de tritons à différentes positions peuvent représenter le même triton. Pour pallier cette difficulté, nous avons pensé à regrouper les images par prise. Les prises ont été séparées entre elles par un intervalle de 2s afin de réduire au plus la probabilité de compter plusieurs fois le même animal. 

\paragraph Une fois les animaux détectés par notre modèle, nous allons procédé au comptage en regroupant les détection par prise. Nous n'allons donc conserver qu'un seul animal de chaque type qui aura été detecté par prise. 

\section{Section1}
\paragraph{paragraph1}
\subsection{Features}

