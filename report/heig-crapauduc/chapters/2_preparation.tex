\chapter{Data preparation}

\section{Acquisition des données}

\paragraph*{Problème} Lors de ce projet, les données doivent être accessible à tous les membres et doivent être stockées de manière uniformisée pour faciliter le travail de groupe. Nous avons alors opté pour une structure regroupant les images par caméra et le nom de fichier correspondant est la date ISO standardisée de la date de la prise du fichier.

\paragraph*{Source} Nous avons récupéré un disque dur comprenant les 500GB dans le bureau de nos professeurs. La structure de fichier était partitionnée par caméra, année, jour, heure, minute. Cette structure était pratique pour naviguer dans les dossiers mais posait un problème pour extraire les informations car les métadonnées étaient stockées dans le path du fichier et non dans un fichier .csv externe. La nouvelle structure partitionnée par camera permet d'avoir toutes les images regroupées et ainsi d'avoir les métadonnées au même endroit. Nous avons ainsi écrit des scripts de transformations que l'on peut trouver dans le repository utils sur github.

\paragraph*{Format} Les images sont au format JPEG.

\paragraph*{Numéro de séquence} Une information qui n'était pas présente originellement était le numéro de séquence des images. Lorsque la caméra détectait un mouvement continu, la même action pouvait résulter sur plusieurs images différentes. Nous avons donc considéré une séquence valide si sur la même caméra, les images sont prise à la suite dans un interval de temps inférieur à 2 secondes. Ce numéro est ainsi ajouté aux métadonnées et permet de réaliser des analyses plus approfondies.

\section{Stockage des Données}

Afin de stocker les données, nous utilisons deux espaces de stockage différents. Premièrement, nous utilisons le serveur atlas mis à disposition pour stocker les images brutes. Deuxièmement, nous utilisons Google Drive pour stocker les subsets d'images traitées. De cette manière, nous avons une source de donnée fiable et pouvons ainsi tous travailler en parallèle avec les mêmes données uniformisées.

\paragraph*{Datalake}

Les données désarborisées ainsi que les données originales sont stockées sur le serveur atlas dans le dossier \texttt{/home/crapauduc/data/}. Ce dossier est accessible à tous les membres du groupe. Les images sont stockées dans des dossiers par caméra et le nom de fichier est la date ISO standardisée de la date de la prise du fichier.

\paragraph{Subsets}

Les subsets sont stockés dans le Google Drive et peuvent être utilisé pour tester//entraîner différents algorithmes



