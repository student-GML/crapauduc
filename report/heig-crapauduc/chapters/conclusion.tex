\chapter{Conclusion}

\paragraph{} En conclusion, on peut estimer que le bilan global de ce projet est en demi-teinte. En effet, nous ne sommes pas parvenus à atteindre tous les objectifs fixés par le cahier des charges. Plus précisément, nous avons atteint seulement deux objectifs parmi les trois "must have" et aucun objectif parmi les trois "nice to have". Ces résultats contrastés sont principalement dûs aux difficultés inattendues rencontrées lors du comptage des animaux. Cependant, l'objectif principal restait de détecter les animaux présents sur les photos ainsi que de différencier les tritons des crapauds-grenouilles. Cet objectif ayant été atteint, non sans difficulté, on peut considérer que le bilan du projet sur le plan technique est plutôt correct.

\paragraph{} Au-delà de l'aspect technique, la gestion de projet représentait également une partie importante de ce projet. Sur ce plan-là, nous pensons tous avoir beaucoup appris au fur et à mesure de l'avancement des différentes étapes du projet. En effet, même si au début la gestion du projet était plutôt chaotique, nous avons réussi à mettre en place des méthodes de travail et de colaboration qui nous ont permis d'avancer de manière plus efficace et qui nous seront probablement utiles pour nos projets futurs et la suite de nos parcours en tant qu'ingénieurs des données. Au vu de cette amélioration au fil du projet, on peut considérer que la gestion de projet est plutôt une réussite même si ce n'était pas gagné d'avance.

\paragraph{} Une conséquence malheureuse de cette entame de projet laborieuse concerne la reproductibilité des expériences effectuées. En effet, comme nous avions tendance à partir dans tous les sens au début du projet, l'organisation des différents fichiers sur le github ou le google drive est relativement chaotique et difficilement compréhensible pour un nouveau venu. Si le projet était à refaire, nous mettrions probablement l'accent sur cet aspect afin de livrer une hiérarchie de fichiers plus lisible et réutilisable par la suite. Malheureusement, nous sommes partis sur de mauvaises bases et ce genre de problème est difficile à résoudre en cours de projet. La reproductibilité du projet représente donc à nos yeux l'un des des échecs de ce projet.

\paragraph{} Finalement, on peut dire que nous avons rencontré un nombre relativement important de difficultés, tant sur le plan technique que sur le plan de la gestion de projet. En effet, nous n'avons pas réussi à atteindre tous les objectifs fixés et la gestion de projet fut parfois extrêmement laborieuse. Cependant, nous avons énormément appris sur ces deux aspects et si le projet était à refaire dans les mêmes conditions, nous obtiendrions probablement des résultats beaucoup plus consistants. Comme il s'agit d'un projet effectué dans une école, la visée est avant tout pédagogique et l'objectif principal est d'apprendre de nos erreurs pour ne pas les reproduire dans le monde du travail. Sur ce point-là, nous sommes donc extrêmement satisfaits car toutes les erreurs effectuées ne se reproduiront plus et nous permettent à toutes et tous d'aller de l'avant et de rejoindre plus sereinement le monde de l'industrie après la fin prochaine de nos études.
